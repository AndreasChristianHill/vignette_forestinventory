\documentclass[article]{jss}

%%%%%%%%%%%%%%%%%%%%%%%%%%%%%%
%% declarations for jss.cls %%%%%%%%%%%%%%%%%%%%%%%%%%%%%%%%%%%%%%%%%%
%%%%%%%%%%%%%%%%%%%%%%%%%%%%%%

%% almost as usual
\author{Andreas Hill \\ETH Z\"urich \And 
        Alexander Massey}
\title{\pkg{forestinventory}: An \proglang{R} Package for Design-Based Global and Small Area Estimations in Multiphase Forest Inventories}

%% for pretty printing and a nice hypersummary also set:
\Plainauthor{Andreas Hill, Alexander Massey} %% comma-separated
\Plaintitle{A Capitalized Title: Something about a Package foo} %% without formatting
\Shorttitle{\pkg{forestinventory}: Design-Based Global and Small Area Estimations} %% a short title (if necessary)

%% an abstract and keywords
\Abstract{
The package provides global- and smallarea estimators for twophase and threephase forest inventories under simple and cluster sampling, which have been developed by Daniel Mandallaz at ETH Zurich. The implemented methods have been published and applied in various studies (see References) and can be used for double- and triple sampling for stratification, double- and triple sampling for regression and double- and triple sampling for regression within strata. 
}
\Keywords{forest inventories, design-based, double and triple sampling, regression estimators, stratification}
\Plainkeywords{forest inventories, design-based, double and triple sampling, regression estimators, stratification} %% without formatting
%% at least one keyword must be supplied

%% publication information
%% NOTE: Typically, this can be left commented and will be filled out by the technical editor
%% \Volume{50}
%% \Issue{9}
%% \Month{June}
%% \Year{2012}
%% \Submitdate{2012-06-04}
%% \Acceptdate{2012-06-04}

%% The address of (at least) one author should be given
%% in the following format:
\Address{
  Andreas Hill\\
  Department of Environmental Systems Science\\
  Chair of Landuse Engineering\\
  ETH Z\"urich\\
  Universit\"atstrasse 16\\
  8092 Z\"urich, Switzerland\\
  E-mail: \email{andreas.hill@usys.ethz.ch}\\
  URL: \url{http://www.lue.ethz.ch/people/hilla}
  
Alexander Massey\\
  Department of Environmental Systems Science\\
  Chair of Landuse Engineering\\
  ETH Z\"urich\\
  Universit\"atstrasse 16\\
  8092 Z\"urich, Switzerland\\
  E-mail: \email{afmass@gmail.com}\\
  URL: \url{http://www.lue.ethz.ch/people/hilla}
}

%% It is also possible to add a telephone and fax number
%% before the e-mail in the following format:
%% Telephone: +43/512/507-7103
%% Fax: +43/512/507-2851

%% for those who use Sweave please include the following line (with % symbols):

%% need no \usepackage{Sweave.sty}

%% end of declarations %%%%%%%%%%%%%%%%%%%%%%%%%%%%%%%%%%%%%%%%%%%%%%%


\begin{document}

%% include your article here, just as usual
%% Note that you should use the \pkg{}, \proglang{} and \code{} commands.


% ---------------------------------- Introduction ---------------------------------------------- %

% --> take parts of the old introduction version of the diploma-thesis!

% NOTES:
% 1st section: - general introduction to forest inventories --> model-'supported' inventories --> 
%                model-dependent vs. design-based
% 2nd section: + (more R-specific): what's already there? --> mention sampling / inventory packages
%                 and seperate them from our package (focus, applied areas, approach)
%              + important reasoning for our package: we are the first that implemented and provide 
%                a large range of design-based estimators
% 3rd section: what is the motivation of the package: - to make the broad range of Mandallaz-estimators
%              available to a large user community, to facilitate their application in science as well as 
%              operational forest management (why is this important? --> because the methods have demonstrate
%              the potential of considerably increasing the estimation precision)
% 4th section: whats the purpose of the paper? --> the objective of the paper is to 
%             a) establish the link between the mathematical description of the estimators (given in Mandallaz ...) 
%                and their implementation in our package
%             b) illustrate the application of the various estimators by the respsective functions in our package 
%                to real-world inventory scenarios
%             c) highlight 'special cases' , i.e. rare inventory scenarios, and demonstrate how the package-functions
%                deal with such situations (error-checking functions and data-adjustments)

% Idee: Reine Theory und Applikation NICHT stark trennen --> Jeweils für jeden Schätzer die nötigsten Formeln geben 
%       (Punktschätzung und g-weight und external variance, wenn nötig noch zusätzliche Infos wie Dummy Variable)
%      WICHITG: Davor muss das generelle Konzept kommen (Entscheidungsbaum und konzeptionelle Erklärung (z.B.: was heisst
%               "exhaustive" usw.) --> auf der Graphik dann ansetzen
%
% evtl auch noch die Art von Graphik für den Aufbau einer 2- und 3-phaisgen Inventurs (wie in den Präsentationen)


% --------------------------- Main structure and modules of the Package ------------------------------- %


% NOTES:
% - Pyramide / decision tree Graphik
% - Confidence Interval calculation
% - Analysis of estimations results. a) Visualization functions; b) Gain-analysis
%
% --> explain each point briefly without to much (mathematical) detail


% ---------------------------------- Estimators and their Application ---------------------------- %

% NOTES:
% 1st subsection: Global 2 and 3-phase estimators for simple and cluster sampling
% 2nd subsection: 2- and 3-phase Small Area estiamtors for simple and cluster sampling
%

% ---------------------------------- Special Cases and Scenarios ---------------------------- %

% NOTES:
% - Error handling and adjustments: Sampling designs not nested
% - Error warning: unbalanced designs and singularities due to missing factor levels in s2-sample (2-phase)
%                   or s2 or/and s1 sample in 3-phase case
% - Error warning: cluster not completely within one small area (--> g-weight variance estimator) +  recommendation


% ---------------------------------- Calculation of Confidence Intervals ----------------------- %

% NOTES:
% - demonstrate calculation of confidence intervals (give the formulas)


% ----------------------------- Visualization and Analysis of Estimation Results ---------------- %

% NOTES:
% - demonstrate visualization of errors and point-ests + CIs
% - demonstrate a) releff-calculation and gain-analysis


% ---------------------------------- Literature ---------------------------- %

\bibliography{bib/literature}\newpage\cleardoublepage

% -------------------------------------------------------------------------- %

\end{document}

