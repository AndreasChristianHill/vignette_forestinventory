\documentclass{article}
\usepackage{authblk}
\usepackage[utf8]{inputenc}
\usepackage[T1]{fontenc}
\usepackage[ngerman]{babel}
\usepackage[normalem]{ulem}

%============================================ Define Titlepage & packages =============================================%

\title{Answer to Editor}

\author{The R Package forestinventory: design-based global and small area estimations for multi-phase forest inventories\\
Andreas Hill, Alexander Massey}

\usepackage{fancyhdr}     
\usepackage{amsmath} %Paket für erweiterte math. Formeln
\usepackage[labelfont=bf]{caption}
\usepackage[font=footnotesize]{caption}
\usepackage[font=footnotesize]{subcaption}
\usepackage{graphicx}
\usepackage{caption}
\usepackage{subcaption}
\usepackage[final]{pdfpages}
\usepackage{color}

\usepackage{geometry}
\geometry{
	a4paper,
	left=25mm,
	right=25mm,
	top=30mm,
	bottom=30mm
}

\setlength{\parindent}{0em} % Einzug bei neuen Absätzen

%------------------------------------------------------------------------------------------------%
% -------------------------------------- Main Document------------------------------------------ %

\usepackage{Sweave}
\begin{document}

%------------------------------------------------------------------------------------------------%
% -------------------------------------- Tex Settings ------------------------------------------ %

\maketitle
\thispagestyle{empty}
\newpage

\pagenumbering{arabic}
\setcounter{page}{1}

\pagestyle{fancy} %Kopfzeile und Fusszeile
\fancyfoot[C]{\thepage}
\setlength{\headsep}{15mm}

\definecolor{mybrown}{rgb}{0.6, 0.15, 0.1}
\definecolor{amaranth}{rgb}{0.9, 0.17, 0.31}
\definecolor{mygreen}{rgb}{0.1, 0.4, 0.4}
\newcommand{\answer}[1]{\small \color{mybrown}{#1} \color{black}}
\newcommand{\note}[1]{\textit{\small \color{amaranth} \textbf{Note:} #1} \color{black}}
\newcommand{\todo}[1]{\color{red}{#1} \color{black}}
\newcommand{\answerfin}[1]{\small \color{mygreen}{#1} \color{black}}
\newcommand{\code}[1]{\texttt{#1}}


%------------------------------------------------------------------------------------------------%
% ---------------------------------- Review -----------------------------------------------------%

\section*{Feedback from Editor and Reviewer A}

Editor:\\
The editor does not understand why the authors were not able to reproduce the code examples given by the reviewer. Is there a bug, or just a misunderstanding between authors on one side and reviewer + editor on the other? Please check the code and explain.\\

Reviewer A:\\
In the revised version of the manuscript (ID 3280), the authors include a new discussion on coverage rates of confidence intervals (p. 25), which, however, appears to only partially embrace the methodological concerns I raised in the earlier review. In particular, the statement that 'the nominal coverage rate of 95\% can [be] expected to be met for large sample sizes (e.g., n2=50)' is in conflict with examples 2 and 3 included from my review (n2=80, coverage rate = 85\%). I am not convinced by the authors' claim that they could not reproduce the examples. On the other hand, the new discussion and in particular Figure 5 are very helpful in illustrating the issue of coverage rates, and I believe this will for most readers be sufficient to make them aware the related problems. Based on my rather short reading of the manuscript and given that a formal review is not required at this stage (after previous recommendation 'Revisions Required'), I would therefore agree that the manuscript can now be accepted.


%------------------------------------------------------------------------------------------------%
% ---------------------------------- Answer -----------------------------------------------------%

\section*{Answer}

The reviewers code runs without bugs but the simulation was incorrectly formulated. One fundamental problem with the reviewers simulation is that he seems to pose all three examples in the \textit{finite} population context rather than the \textit{infinite} population context. We can see this right at the beginning:\\

Here, the reviewer sets up a dataframe defining a \textit{finite} population that he then uses in his simulation examples:

\begin{small}
\begin{Schunk}
\begin{Sinput}
> # Reviewer's artificial data:
> set.seed(0)
> d <- data.frame(
+           elv = (e<-runif(500, 300, 550)),     # explanatory variable (observed in s1)
+           vol1 = rnorm(e, mean=600-e, sd=100), # response variables (to be observed in s2 only)
+           vol2 = rnorm(e, mean=300, sd=100)*(e<310),
+           resp3 = runif(e) < 0.03)
\end{Sinput}
\end{Schunk}
\end{small}

Let's look at the first simulation example to explain where the fundamental problem is:

\begin{small}
\begin{Schunk}
\begin{Sinput}
> library(forestinventory)
> # PROBLEM 1: Small sample size (n2=5)
> true_value <- mean(d$vol1)
> cover <- replicate(10000, {
+ d$phase <- 1
+ d$phase[sample(length(d$phase), 5)] <- 2
+ ci <- confint( twophase( vol1 ~ elv, data = d,
+ phase_id = list(phase.col="phase", terrgrid.id=2) ), level=0.95 )
+ ci$ci$ci_lower_g <= true_value & true_value <= ci$ci$ci_upper_g
+ } )
> mean(cover)
\end{Sinput}
\end{Schunk}
\end{small}

As obvious, the true population mean (\code{true\_value}) that has to be estimated is here calculated by taking the mean of a \textit{finite} number of response variable values ($N=500$, i.e. \code{nrow(d)}). This shows that the reviewer actually defined a finite population for the response variable (\code{vol1}), and likewise for the auxiliary variable (\code{elv}). In every iteration of his simulation, he then resamples from a finite set of observed response variable values $s_2$. This is a fundamental misunderstanding of the infinite population theory, which was exclusively used to derive all estimators contained in \code{forestinventory}. First of all, the true population mean (\code{true\_value}) should be an integral (i.e. spatial mean) over infinite points, as described in section 2.1 in the article, but by using the finite mean, the reviewer has effectively acknowledged that actually $N=500$ instead of infinity.\par
Second, we then see that every population unit of his finite population is assigned to the first phase (\code{d\$phase<- 1}). Basically, this means that the auxiliary information is not resampled in every simulation iteration, but the entire population ($N=500$) of auxiliary information \code{elv} is used. This indicates that the reviewer intended to apply an \textit{exhaustive} two-phase estimator where the true population mean of the auxiliary information is used. Notice that in the \code{twophase()}-function call, the reviewer, however, did \textit{not} set the exhaustive parameter as demonstrated in section 3.1 pp. 11-12 in the article, making this a non-exhaustive function call. This means that by simulation design $s_1$ does not get resampled with every replicate so the variability coming from resampling the first phase is ignored.\\

All 3 examples given by the reviewer made these same errors so the results clearly can't be considered conclusive on their own merits.  This is why we attempted to reformulate the simulation examples correctly in the infinite population context and mimic the examples intended effects. The following is some sample code (similar code was added to the Appendix in the last submission) to demonstrate simulation in the infinite population context.  This technique has already applied in Mandallaz (2013a), Mandallaz et al. (2013) and Massey et al. (2015):


\begin{small}
\begin{Schunk}
\begin{Sinput}
> library(forestinventory)
> library(rmutil)  # used to calculate the integral in the spatial mean
> #
> #
> set.seed(1)
> #
> # --- function to create density surface: --- #
> target.surface <- function(x0, y0){
+   local.density <- 30 + 13*x0 - 6*y0 - 4*x0^2 + 3*x0*y0 + 2*y0^2 + 6*cos(pi*x0)*sin(pi*y0)
+   local.density
+ }
> #
> # --- True spatial mean for Global Surface Area: --- #
> true_value <- int2(target.surface, a = c(0,0), b=c(3,2))/6 # 39.16667  #true mean
> #
> # --- sample generator for simple two-phase sampling: --- #
> sample.generator<- function(n1, n2){
+   
+   realization <- matrix(NA,n1,6)
+   for(i in 1:n1){
+     x0=2*runif(1) # NOTE: The randomness comes from the uniform independent selection of points
+     y0=3*runif(1)
+     realization[i,1] <- target.surface(x0, y0)
+     realization[i,2] <- x0
+     realization[i,3] <- y0
+     realization[i,4] <- x0*x0
+     realization[i,5] <- x0*y0
+     realization[i,6] <- y0*y0
+   }
+   realization <- as.data.frame(realization)
+   names(realization) <- c("response","x","y","xx","xy","yy")
+   realization$phase <- 1
+   realization$phase[sample(nrow(realization), n2)] <- 2
+   realization$response[realization$phase == 1] <- NA
+   realization
+ }
\end{Sinput}
\end{Schunk}
\end{small}

As obvious, we take a finite random sample $s_2$ and $s_1$ from an infinite number of points (note that the number of possible points, i.e., coordinate pairs \code{x0, y0}) as input for the density function \code{target.surface} is infinite). The randomness in the sampling procedure comes from the uniform independent selection of points that are randomly distributed over the density surface defining the distribution of the target variable at every possible point in the plane.


\subsection*{Reviewer Problem \#1}

\begin{small}
\begin{Schunk}
\begin{Sinput}
> # --- Run 10000 simulations with n1 = 500 and n2 = 5 (small sample size n2) --- #
> n1 <- 500
> n2 <- 5
> num_sims <- 10000
> ci.g.logical <- rep(NA, num_sims)   # two-phase coverage indicators for g-weight formula
> ci.ext.logical <- rep(NA, num_sims) # two-phase coverage indicators for external formula
> ci.op.logical <- rep(NA, num_sims)  # one-phase coverage indicators
> #
> for(j in 1: num_sims){
+   realization <- sample.generator(n1 = n1, n2 = n2)
+   est <- twophase(formula = response ~ y + x + xx, data=realization, 
+                   phase_id = list(phase.col = "phase", terrgrid.id = 2))
+   est_onephase <- onephase(formula = response ~ 1, data=realization, 
+                            phase_id = list(phase.col = "phase", terrgrid.id = 2))
+   ci <- confint(est)
+   ci_op <- confint(est_onephase)
+   ci.g.logical[j] <- ci$ci$ci_lower_g <= true_value & true_value <= ci$ci$ci_upper_g
+   ci.op.logical[j] <- ci_op$ci$ci_lower_op <= true_value & true_value <= ci_op$ci$ci_upper_op
+ }
> #
> # results for coverage rates:
> mean(ci.g.logical)   # 0.6738, g-weight coverage rate of 2-phase estimator
> mean(ci.op.logical)  # 0.9457, coverage rate of one-phase estimator
\end{Sinput}
\end{Schunk}
\end{small}

The one-phase estimator has the correct coverage rate even with sample size 5 (it would also have it for $n_2$ = 2) but it will probably have a very large margin of variance (i.e. estimation error). The g-weight variance has lower than nominal coverage rates because the contribution of the variance by calculating the beta vector is based on approximations (i.e. a first order taylor approximation for the g-weight) which are not adequate for this sample size.  The article has already included a graphic with sample code that addresses the sample size issue and the reviewer has already accepted it so we can leave this issue alone for the time being.


\subsection*{Reviewer Problem \#2 and \#3}

The 2nd reviewer example involves creating a response variable \code{vol2}, that is \code{0} for 98.2\% of the values and non-zero where the auxiliary information yields a low value.  Presumably this would occur in situations where there is a treeline dependent on elevation and the vast majority of the inventory domain is at elevations where no trees can grow.  This is problematic for the g-weight and external estimators because all plots in $s_2$ where trees actually exist are high leverage outliers. However, a quick look at the reported R-squared, which would be very close to zero, should immediately trigger the  question why the model is included in the first place.\\

Problem 3 is actually the exact same example except that the response variable takes exclusively the value \code{1} at non-zero plots and these non-zero plots (and thus plots with response variable value=1) are uniformly distributed among the auxiliary variables. This means that there is per-definition no causal correlation between the auxiliary variable values and the response value, which arises the question why one should use the auxiliary information at all. Nevertheless, such a situation may occur if say the inventorist wanted to count how many trees are in the desert where there are just a handful of uniformly distributed oases.  Even in this case, the nonzero plots will again be highly influential leverage points in the presence of so many zeros. This issue is easily seen by low R-squared values as well as by a look at the residuals of the linear model.\\

These two examples are also pathological for the one-phase estimator because a significant proportion of the samples drawn will have zeros for ALL of s2 making the estimate = 0 and the confidence interval just a point with no chance of covering the true value.  If this occurred in practice the sample would immediately be considered suspect.  It is hard to imagine a situation where this would accidentally happen.  Furthermore, trying to measure forest volume in areas where 98\% of the land has no trees is very unlikely and would be difficult to accidentally do.


\end{document}
























